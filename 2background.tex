\chapter{Background}
\label{chapter:background} 

\fixme{Needs a less generalized title.}

\section{Threat Analysis}

\fixme{\begin{itemize}
\item Some history on threat analysis
\item Research on the importance of threat analysis
\item Some mistakes made in the past, as an example
\item Previous research done on different threat analysis methods
\end{itemize}}


\newpage
\fixme{I'll write about "early history" when I find sources.}


\citet{threat_analysis_1993} separates the different threat analysis methods used up until 1993 into three generations: the first generation starting from 1972, the second generation from 1981 onward and the third generation that was introduced in 1988.

The first generation relies heavily on check-lists composed by authorities in the field. 
The systems these first-generation methods were used on were much simpler than the ones in use now, and security analysis consisted of choosing the best option of a limited list of known components, instead of the wealth of options that developers currently have.
They do not expect the analyst to have deep knowledge, as independent analysis is not needed.
It was also more focused on hardware than software. \cite{threat_analysis_1993}

The second generation came when the systems got too complicated for the first generation's check-list method. 
It relies on partitioning the system into smaller components and then coming up with a solution that matches the functional requirements of each component.
Secong generation methods are more complex, and the analysts need a higher degree of training, but does not rely on a set solution set and can be used in much more complex systems. \cite{threat_analysis_1993}

Third generation relies on more abstraction compared to the second generation. 
Instead of partitioning the system into components like int second generation, the third generation relies on building abstract models of the systems, and using them as an aid in the analysis.
These methods are most useful when designing the system, and the amount of training is even higher than in second generation.
On the other hand, third generation solutions are more flexible and should lead to less conflict between security and usability. \cite{threat_analysis_1993}

\citet{risk_analysis_to_security_requirements} also divide the types of risk analysis done in the past into three different time periods.

The first era that they call the Computer-centric era is before the early 1980's, corresponding to a time before what \citet{threat_analysis_1993} calls the first generation.
According to \citet{risk_analysis_to_security_requirements}, a company's business didn't depend on computers and computer security then as it does now.
Assets were easy to protect, as the protection could be done with physical controls like locks on the doors and threats could be found using simple check lists.

The second era, or the IT-centric era, lasted from the early 1980's to the early 1990's, which corresponds to what \citep{threat_analysis_1993} called the first and second generations.
The businesses were increasingly dependent on computers, and security became more and more important.
The identifying of the assets became an issue, as they were not physically located in the same place any more, and sometimes were not even physical.
This lead to first using impact values, which depend on subjective guesses on impacts of threats, and then to qualitative techniques using tables with impact values and the probability of threats. \citet{risk_analysis_to_security_requirements}

The third era, called the Information-Centric era, started in the 1990's.
In the Information-Centric era businesses depend on information and computers, and risk analysis is no longer enough as legal and business requirements become more and more important.
This has lead to a trend of moving away from pure risk analysis, and into mixed risk and security requirements analysis, taking into account the unique risks and requirements each system and business has. \citet{risk_analysis_to_security_requirements}

\newpage

\section{Authentication and Authorization systems}

\fixme{\begin{itemize}
\item Some history again
\item Common flows in systems
\item Examples of different kinds
\item Importance of good system and its security
\end{itemize}}