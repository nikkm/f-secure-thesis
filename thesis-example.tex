\documentclass[12pt,a4paper,oneside,pdftex]{report}

\usepackage[latin1]{inputenc}
\usepackage[OT1]{fontenc}
\usepackage[finnish,swedish,english]{babel}
% \usepackage{palatino}
% \usepackage{tgpagella}


% Optional packages
% ------------------------------------------------------------------

% Natbib allows you to select the format of the bibliography references.
% The first example uses numbered citations: 
\usepackage[square,sort&compress,numbers]{natbib}
% Also, you should use \citet (cite in text) when you wish to refer to the author directly (\citet{blaablaa} said blaa blaa), and \citep when you wish to refer similarly than with numbered citations
% (It has been said that blaa blaa~\citep{blaablaa}).

% The eurosym package provides a euro symbol. Use with \euro{}
%\usepackage{eurosym} 

% The listing package provides automatic code formatting utilities so that you can copy-paste code examples and have them rendered nicely.
% See the package documentation for details.
\usepackage{listings}

% Supertabular provides a tabular environment that can span multiple pages. 
%\usepackage{supertabular}
% Longtable provides a tabular environment that can span multiple pages.
% This is used in the example acronyms file. 
\usepackage{longtable}

% The fancyhdr package allows you to set your the page headers manually, and allows you to add separator lines and so on. 
% Check the package documentation. 
% \usepackage{fancyhdr}

% Subfigure package allows you to use subfigures (i.e. many subfigures within one figure environment).
% These can have different labels and they are numbered automatically.
% Check the package documentation. 
\usepackage{subfigure}

% The titlesec package can be used to alter the look of the titles of sections, chapters, and so on.
% This example uses the ``medium'' package option which sets the titles to a medium size, making them a bit smaller than what is the default.
% You can fine-tune the title fonts and sizes by using the package options.
% See the package documentation.
\usepackage[medium]{titlesec}

% The TikZ package allows you to create professional technical figures.
% The learning curve is quite steep, but it is definitely worth it if  you wish to have really good-looking technical figures. 
\usepackage{tikz}
% You also need to specify which TikZ libraries you use
\usetikzlibrary{positioning}
\usetikzlibrary{calc}
\usetikzlibrary{arrows}
\usetikzlibrary{decorations.pathmorphing,decorations.markings}
\usetikzlibrary{shapes}
\usetikzlibrary{patterns}


% Aalto-thesis
% ------------------------------------------------------------------

% The aalto-thesis package provides typesetting instructions for the standard master's thesis parts (abstracts, front page, and so on).
% Load this package second-to-last, just before the hyperref package.
% Options that you can use: 
%   mydraft - renders the thesis in draft mode. 
%             Do not use for the final version. 
%   doublenumbering - [optional] number the first pages of the thesis
%                     with roman numerals (i, ii, iii, ...); and start
%                     arabic numbering (1, 2, 3, ...) only on the 
%                     first page of the first chapter
%   twoinstructors  - changes the title of instructors to plural form
%   twosupervisors  - changes the title of supervisors to plural form
\usepackage[mydraft,twosupervisors]{aalto-thesis}
%\usepackage[mydraft,doublenumbering]{aalto-thesis}
%\usepackage{aalto-thesis}


% Hyperref
% ------------------------------------------------------------------
% Hyperref creates links from URLs, for references, and creates a TOC in the PDF file.
% This package must be the last one you include, because it has compatibility issues with many other packages and it fixes those issues when it is loaded.   
\RequirePackage[pdftex]{hyperref}
% Setup hyperref so that links are clickable but do not look different
\hypersetup{colorlinks=false,raiselinks=false,breaklinks=true}
\hypersetup{pdfborder={0 0 0}}
\hypersetup{bookmarksnumbered=true}
% The following line suggests the PDF reader that it should show the first level of bookmarks opened in the hierarchical bookmark view. 
\hypersetup{bookmarksopen=true,bookmarksopenlevel=1}
% Hyperref can also set up the PDF metadata fields. 
% These are set a bit later on, after the thesis setup.   


% Thesis setup
% ==================================================================
% Change these to fit your own thesis.
% \COMMAND always refers to the English version;
% \FCOMMAND refers to the Finnish version; and
% \SCOMMAND refers to the Swedish version.
% You may comment/remove those language variants that you do not use (but then you must not include the abstracts for that language)
% ------------------------------------------------------------------
% If you do not find the command for a text that is shown in the cover page or in the abstract texts, check the aalto-thesis.sty file and locate the text from there. 
% All the texts are configured in language-specific blocks (lots of commands that look like this: \renewcommand{\ATCITY}{Espoo}.
% You can just fix the texts there.
% Just remember to check all the language variants you use (they are all there in the same place). 
% ------------------------------------------------------------------
\newcommand{\TITLE}{English Name PLACE-HOLDER}
\newcommand{\FTITLE}{Finnish Name PLACE-HOLDER}
\newcommand{\SUBTITLE}{English Subtitle PLACE-HOLDER}
\newcommand{\FSUBTITLE}{Finnish Subtitle PLACE-HOLDER}
\newcommand{\DATE}{English Date PLACE-HOLDER}
\newcommand{\FDATE}{Finnish Date PLACE-HOLDER}


% Supervisors and instructors
% ------------------------------------------------------------------
% Example of one supervisor:
\newcommand{\SUPERVISOR}{English Supervisor PLACE-HOLDER}
\newcommand{\FSUPERVISOR}{Finnish Supervisor PLACE-HOLDER}

% If you have only one instructor, just write one name here
\newcommand{\INSTRUCTOR}{English Instructor PLACE-HOLDER}
\newcommand{\FINSTRUCTOR}{Finnish Instructor PLACE-HOLDER}


% Other stuff
% ------------------------------------------------------------------
\newcommand{\PROFESSORSHIP}{English Professorship PLACE-HOLDER}
\newcommand{\FPROFESSORSHIP}{Finnish Professorship PLACE-HOLDER}
% Professorship code is the same in all languages
\newcommand{\PROFCODE}{Code PLACE-HOLDER}
\newcommand{\KEYWORDS}{English Keywords PLACE-HOLDER}
\newcommand{\FKEYWORDS}{Finnish Keywords PLACE-HOLDER}
\newcommand{\LANGUAGE}{English}
\newcommand{\FLANGUAGE}{Englanti}

% Author is the same for all languages
\newcommand{\AUTHOR}{Mari Nikkarinen}

% Currently the English versions are used for the PDF file metadata
% Set the PDF title
\hypersetup{pdftitle={\TITLE\ \SUBTITLE}}
% Set the PDF author
\hypersetup{pdfauthor={\AUTHOR}}
% Set the PDF keywords
\hypersetup{pdfkeywords={\KEYWORDS}}
% Set the PDF subject
\hypersetup{pdfsubject={Master's Thesis}}


% Layout settings
% ------------------------------------------------------------------

% If you write your thesis Finnish, uncomment these lines; if you write in English, leave these lines commented! 
% \setlength{\parindent}{0pt}
% \setlength{\parskip}{1ex}

% Use this to control how much space there is between each line of text.
% 1 is normal (no extra space), 1.3 is about one-half more space, and 1.6 is about double line spacing.  
% \linespread{1} % This is the default
% \linespread{1.3}

% Bibliography style
% acm style gives you a basic reference style.
% It works only with numbered references.
%\bibliographystyle{acm}
% Plainnat is a plain style that works with both numbered and name citations.
\bibliographystyle{plainnat}


% Extra hyphenation settings
% ------------------------------------------------------------------
% You can list here all the files that are not hyphenated correctly.
% You can provide many \hyphenation commands and/or separate each word with a space inside a single command.
% Put hyphens in the places where a word can be hyphenated.
% Note that (by default) LaTeX will not hyphenate words that already have a hyphen in them (for example, if you write ``structure-modification operation'', the word structure-modification will never be hyphenated).
% You need a special package to hyphenate those words.
\hyphenation{di-gi-taa-li-sta yksi-suun-tai-sta}



% The preamble ends here, and the document begins. 
% Place all formatting commands and such before this line.
% ------------------------------------------------------------------
\begin{document}
% This command adds a PDF bookmark to the cover page.
% You may leave it out if you don't like it...
\pdfbookmark[0]{Cover page}{bookmark.0.cover}
% This command is defined in aalto-thesis.sty.
% It controls the page numbering based on whether the doublenumbering option is specified
\startcoverpage


% Cover page
% ------------------------------------------------------------------
% Options: finnish, english, and swedish
% These control in which language the cover-page information is shown
\coverpage{english}

% Abstra
cts
% ------------------------------------------------------------------
% Include an abstract in the language that the thesis is written in, and if your native language is Finnish or Swedish, one in that language.


% Abstract in English
% ------------------------------------------------------------------
\thesisabstract{english}{

\fixme{The abstract is written last.}}

% Abstract in Finnish
% ------------------------------------------------------------------
\thesisabstract{finnish}{
\fixme{Abstrakti kirjoitetaan viimeisen�.}}

% Acknowledgements
% ------------------------------------------------------------------
% Select the language you use in your acknowledgements
\selectlanguage{english}

% Uncomment this line if you wish acknoledgements to appear in the table of contents
%\addcontentsline{toc}{chapter}{Acknowledgements}

% The star means that the chapter isn't numbered and does not show up in the TOC
\chapter*{Acknowledgements}

\fixme{Acknowledge some people.}

\vskip 10mm

\noindent Espoo, \DATE
\vskip 5mm
\noindent\AUTHOR

% Acronyms
% ------------------------------------------------------------------
% Use \cleardoublepage so that IF two-sided printing is used (which is not often for masters theses), then the pages will still start correctly on the right-hand side.
\cleardoublepage
% Example acronyms are placed in a separate file, acronyms.tex
\input{acronyms}

% Table of contents
% ------------------------------------------------------------------
\cleardoublepage
% This command adds a PDF bookmark that links to the contents.
% You can use \addcontentsline{} as well, but that also adds contents entry to the table of contents, which is kind of redundant.
% The text ``Contents'' is shown in the PDF bookmark. 
\pdfbookmark[0]{Contents}{bookmark.0.contents}
\tableofcontents

% List of tables
% ------------------------------------------------------------------
% You only need a list of tables for your thesis if you have very many tables.
% If you do, uncomment the following two lines.
% \cleardoublepage
% \listoftables

% Table of figures
% ------------------------------------------------------------------
% You only need a list of figures for your thesis if you have very many figures.
% If you do, uncomment the following two lines.
% \cleardoublepage
% \listoffigures

% The following label is used for counting the prelude pages
\label{pages-prelude}
\cleardoublepage

%%%%%%%%%%%%%%%%% The main content starts here %%%%%%%%%%%%%%%%%%%%%
% ------------------------------------------------------------------
% This command is defined in aalto-thesis.sty.
% It controls the page numbering based on whether the doublenumbering option is specified
\startfirstchapter

% Add headings to pages (the chapter title is shown)
\pagestyle{headings}

% The contents of the thesis are separated to their own files.
% Edit the content in these files, rename them as necessary.
% ------------------------------------------------------------------
\chapter{Introduction}
\label{chapter:intro}

\fixme{This is a good place to start the writing.}


\chapter{Background}
\label{chapter:background} 

\fixme{This needs a more fitting title, but this'll do for now.
Background on what I'm doing needs to be written.}

\chapter{Environment}
\label{chapter:environment}

\fixme{\begin{itemize}
\item OneID explanation goes here
\item Scrubbed of everything secret
\end{itemize}}


\chapter{Methods}
\label{chapter:methods}

\fixme{How and why the analysis is done goes here.}

 
\chapter{Implementation}
\label{chapter:implementation}

\fixme{This might need a better title too.
How did I do it goes here.}

\input{6evaluation.tex}
 
\chapter{Discussion}
\label{chapter:discussion}

\fixme{Insights about the work belongs here.}

 
\chapter{Conclusions}
\label{chapter:conclusions}

\fixme{Wrap-up here.
Basically the whole thing in a nutshell.
Written at the end.}


% Load the bibliographic references
% ------------------------------------------------------------------
% You can use several .bib files:
% \bibliography{thesis_sources,ietf_sources}
\bibliography{sources}


% Appendices go here
% ------------------------------------------------------------------
% If you do not have appendices, comment out the following lines
\appendix
\chapter{First appendix}
\label{chapter:first-appendix}

\fixme{Any appendices here.}


% End of document!
% ------------------------------------------------------------------
% The LastPage package automatically places a label on the last page.
% That works better than placing a label here manually, because the label might not go to the actual last page, if LaTeX needs to place floats (that is, figures, tables, and such) to the end of the document.
\end{document}
